\chapter{X-Style}
Competitors only care about skills.
This system finds at least the best 3 riders.

\section{Definitions}
\textbf{Unicycle Skill}: The main action of a movement is done on and with the unicycle.

\emph{Example 1}: If you do a handstand, while the unicycle is laying on the floor, this is a non-unicycle trick, because the main action is the handstand.

\emph{Example 2}: If you do a handstand on the unicycle, while it is standing upside down, this is a unicycle trick, because the main action is balancing the unicycle while doing the handstand.

\section{Runs}
Every competitor has a run in which the rider shows his/her skills.
A DJ plays music for each run.
After the run the DJ waits until the current skill is finished and stops the music.
Riders can also hand in their own music.
Riders can ask the DJ to give a time-remaining notification during his/her run.

\subsection{Run Length}
The event director can choose the length of the runs, but it must be announced at least one month before the competition if it differs from the recommended format below.
Runs must be between 1 and 2 minutes 30 seconds.

This is the recommended format: \\
The length of a competitor's run is determined by the round.
\begin{itemize}
\item If three or more rounds: \\
1st round: 1 minute \\
Intermediate rounds: 1 minute 30 seconds \\
Finals: 2 minutes.
\item If two rounds: \\
1st round: 1 minute 30 seconds \\
Finals: 2 minutes
\item If one round: 2 minutes
\end{itemize}

\section{Starting Groups}
The group of competing riders is split randomly into starting groups of about the same size with a maximum 10 riders.
There can be two distinct tournaments for junior (age 14 or younger) and senior (age 15 and older) riders.
If there are obvious critical groups (e.g. all the best riders in one group), the chief judge is allowed to modify the groups.

The host can decide to order the riders by age and then split them into starting groups.
The host is also allowed to hand out awards to intermediate winners.
This can be motivating for younger riders.

\section{Tournament}
The best 3 riders of each group advance to the next round, forming a new pool of competitors.
If there is a tie on one of the first 3 places which extends to more than the 3 riders, all involved riders advance.
This pool again gets split into starting groups and the next round begins.
The final round consists of only one starting group.
In the final round the best 3 riders are awarded for the first, second and third place of the competition.

\subsection{Ties}
If the competition doesn't allow ties (e.g. Unicon), the tied placed riders of the final round will be given another run.
The run will be one minute.
The judges bring all tied riders in order.
If there is still a tie between riders, the tie rules will be applied again, until all ties are resolved.

\section{Judges}
The Chief Judge composes the Judging Table for each starting group.
All judges can be active competitors or non-competitors.
Preferred are people with judging experience and competitors.
Non-Competitors can apply for being a judge by contacting the Chief Judge in advance.
The Chief Judge sets the application deadline.
It is recommended that every starting group is judged by two other starting groups.
The judging table consists of at least 5 Judges.

\section{Judging}
Every routine is judged by the judging table.
One judging table stays for one starting group.
Judges can judge alone or in pairs.
Judging in pairs is the preferred system.
All judges must either judge alone or in pairs so that each judge's vote has equal weight.
(Pair judges are referred to as one judge below.)
The judge should rank the riders of the current starting group in order.
They should do this by comparing the difficulty of the shown skills.
The same skill when completed with higher quality (for example elegant, smooth, or clean) is considered more difficult.
Assigning the same rank to multiple riders is allowed.

Only executed skills are taken into account.
An executed skill is defined as when the rider reaches the point of being in control.

Examples:
\begin{itemize}
\item The landing of a unispin is part of the skill.
The rider can only reach the point of being in control after landing.
If the rider is hopping four times after the unispin without control and then falls off the unicycle, the skill does not count.
\item In coasting, the rider is in control after getting far enough.
Getting back to pedals is a separate skill.
\end{itemize}

Negative aspects like dismounts are ignored.
Every judge should use blank sheets of paper to take notes.

The highest and the lowest placing points per rider are discarded.
All the remaining placing points get summed up for each rider.
The 3 riders with the fewest points win and advance to the next round.

\section{Publishing Results}
The published results contain the riders total ranking in order with their summed placing points and the anonomized results of the judges with their ranking for each rider.
