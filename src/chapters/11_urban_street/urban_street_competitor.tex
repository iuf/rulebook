\chapter{Competitor Rules}

\section{Safety}

\oldrule{6.3}
Riders must wear shoes, shin pads, and helmet.

\section{Unicycles}

\oldrule{6.4}
Standard unicycles only (see definition in chapter \ref{chap:general_definitions}), though any number can be used.
Unicycles with metal pedals and marking tires are allowed, so these competitions are generally intended for outdoors.

\section{Rider Identification}

\begin{framed}
Does the rider need to wear any identification?
\end{framed}

\section{Music, Costume and Props}
\oldrule{6.8}
In Flatland, competitors may optionally bring their own music but is not judged.

\subsection{Media Types}
\oldrule{6.8.1}
The host is required to have the capability of playing recordable CDs.
Other media types may also be supported, at the host's discretion.

\subsection{Music Preparation}
\oldrule{6.8.2}
Competitors who bring music must provide it in a form that is supported, and has been announced by the Artistic Director.
All music must be clearly labeled with the competitor name, age group and the track number.
Whenever possible, competition music should be the first track on the CD.
The DJ (music operator) is not responsible for any errors resulting from unsupported types or mislabeled tracks.

\subsection{Costume and Props}
\oldrule{6.8.3}
Clothing has no influence on the score.
Riders are encouraged to dress in the uniform of their national teams or clubs, or in clothing that represents their teams, groups or countries.
No props allowed, other than what is included in the performing area.

\section{Protests}

\oldrule{6.10}
Must be filed in writing, within 15 minutes from the posting of event results.
Protest against judges' scores is not permissible.
Protest is only possible against calculation mistakes or other mistakes not connected to the scoring.

\section{Event Flow}

\subsection{Deadline For Entry}

\oldrule{6.2}
\textbf{This event has} a deadline for entry, which must be specified in the registration form.
If not specified in the registration form, the deadline is one month before the official convention start date.

\subsection{Riders Must Be Ready}

\oldrule{6.6}
Riders who are not ready at their scheduled performance time may or may not be allowed to perform after the last competitor in their age group.

\section{Preliminaries}
\oldrule{6.29}
Riders will be put into groups of three or four (preferably 4, but in some cases, there may need to be up to 3 groups of 3 depending on the number of competitors).
Each group will be given a starting time, and they will proceed to their starting Zone.
They will be given 5 minutes in each zone to perform as many tricks as possible.
The riders are assigned an order and they may only attempt a trick when it is their turn.
The order should be presented in writing as well as announced before the competition.
Riders may choose to skip their turn in the event of an injury or any other reason.
The group will then move on to the next zone (so it will take each group 25 minutes to finish, with 5 minutes after for discussion, and it will take 10n+20 minutes to finish prelims, where n is the number of groups).

\section{Finals}
\oldrule{6.30}
The top 5 or 6 riders will be chosen to participate in the finals, which should be a few hours later, or the next day.
Finals should preferably not be before noon, because we want a lot of spectators, and we want to riders to have a chance to warm up and be ready to be at their best.
In the finals, the same 3 zones will be used, and all riders will go at the same time for 12 to 15 minutes (open for discussion) in each zone.
The riders are assigned an order and they may only attempt a trick when it is their turn.
The order should be presented in writing as well as announced before the competition.
Riders may choose to skip their turn in the event of an injury or any other reason.
There will be 5 judges in the finals, and these can be made up from some of the judges of prelims, or even riders that may not have performed their best in prelims, and did not make it into the finals.
