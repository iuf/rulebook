\chapter{Competitor Rules}

\section{Safety}

Riders must wear shoes, knee pads and gloves (definitions in chapter \ref{chap:general_definitions}), unless otherwise noted, below.

Riders on wheels larger than 24 Class (or with gearing) must also wear helmets.

The Downhill Coast, Downhill Glide, and 50m Fast Backward races require helmets.

\section{Unicycles}

Only regular unicycles may be used.
Riders may use different unicycles for different racing events, as long as all comply with the rules for events in which they are entered.

For events divided by wheel size, there is a maximum allowable tire diameter and minimum crank arm length for each category:

\begin{longtable}{|p{3cm}|p{3cm}|p{4cm}|p{3cm}|}
\hline
\textbf{Unicycle Class} & \textbf{Max Diameter} & \textbf{Min Crank Length} & \textbf{Transmission}\\
\hline
16 Class & 418mm & 89mm & regular \\
\hline
20 Class & 518mm & 100mm & regular \\
\hline
24 Class & 618mm & 125mm & regular \\
\hline
29 Class & 778mm & No limit & regular \\
\hline
Unlimited Class & No limit & No limit & unlimited \\
\hline
\end{longtable}

For any tire in question, its outside diameter must be accurately measured.

Crank arm length is measured from the center of the wheel axle to the center of the pedal axle.
Longer sizes may be used.

In all track racing events on regular unicycles, shoes must not be fixed to the pedals in any way (no click-in pedals, toe clips, tape, magnets or similar).

\section{Rider Identification}

Riders must wear their race number clearly visible on their chest so that it is visible during the race and as the rider crosses the finish line.
Additionally, the rider may be required to wear a chip for electronic timing.

\section{Protests}

Protests must be filed on an official form.
Mistakes in paperwork, inaccuracies in placing, and interference from other riders or other sources are all grounds for protests.
All Referee decisions are final, and cannot be protested.
For a large event such as Unicon or continental championships, the default protest time is 60 minutes (counting from the posting of results), the minimum is 30 minutes.
For smaller events, the default protest time is 30 minutes, the minimum is 15 minutes.
Every deviation from the default protest time has to be clearly announced when the results are posted, including stating the protest deadline on the results list itself.
The protest time may be extended for riders who have to be in other races during the protest period.
All protests will be acknowledged within 30 minutes from the time they are received, and an effort will be made to settle the issue within those 30 minutes.

\section{Wheel Size Categories}

Wheel sizes for track racing are 20 Class, 24 Class and 29 Class.
Additional groups for 16 Class or other wheels can be added.
When not otherwise specified, 24 Class is the maximum wheel size above age 10.
For age groups with a maximum age of 10 or younger, the maximum wheel size is 20 Class (or smaller, if smaller sizes are also used).
The youngest age group for 24 Class wheels should have a minimum age of 0, so riders 10 and younger have the option of racing on 24 Class with those groups (e.g.\ 0-13 or 14-16).

\section{Event Flow}

\begin{comment2016}
  A lot of work needs to be done, if these event rules are going to be formalized.
  For now, they can borrow from Track, and an event organizer needs to specify the rules they want to use.
\end{comment2016}

In general, the rules of Track apply, such as false starts, lane use, and dismounts.

\subsection{Coasting Events}
An event to determine which rider coasts the furthest distance.
Riders' coasting distances are measured from a `starting line' with a 5 meter minimum, which will be marked by a `qualifying line.'
If the rider does not cross the qualifying line it will count as a failed attempt.
The farthest distance from the line wins.
The distance is measured to the rearmost part of the rider that touches the ground when dismounting, or to the tire contact point where the rider stops coasting.
Remounting is not allowed.
Riders get two attempts.
If a rider crosses the coasting line (tire contact point) not in coasting position, he or she is disqualified in that attempt.
The riding surface should be as smooth and clean as possible, and it may be straight or curved.
Ample time must be allowed for all competitors to make some practice runs on the course before the official start.
The type of event(s) to be used should be announced well in advance of the competition.
Crank arm rules do not apply in any coasting or gliding events.

\subsubsection{Road Coasting}

This event is best held on a roadway with a very slight downward slope.
Riders are allowed an unlimited distance to speed up and start coasting before the starting line.

\subsubsection{Track Coasting \label{subsubsec:track-field_alternate-optional-fun-events_coasting_track-coasting}}
30 meter speed-up distance.
This event is held only on a track, or a very level, smooth surface.
Wind must be at a minimum for records to be set and broken.
This event can be compared with other races at different tracks worldwide.

\subsubsection{Downhill Coasting}
This is a speed coasting event.
Riders start from a standstill, or speed up to the `starting line'.
Riders are timed over a measured distance to the finish line.
Dismounts before the finish line disqualify the rider in that attempt.
The slope must be very gradual for this event to be safe, and helmets are mandatory.

\subsubsection{Indoor Coasting}
30 meter starting distance.
This event is held indoors in a gym, or on a very level, smooth surface.
Rider will coast in a circle on the outer edge of the gym, separated by cones.
Both directions are allowed for the start (clockwise or counterclockwise), and rider will have a maximum of 30m before beginning to coast.
Indoor coasting is the recommended coasting competition at a Unicon.

\subsection{Gliding Events}

In Gliding, the balance has to be kept all the time by the braking action between one or both feet and the top of the tire.
If, for example, the foot loses contact with the tire due to small bumps, the contact must be restored immediately.

Gliding events are similar to the coasting events above, with riders gliding for time or distance from a given point.
The rules are the same as for the coasting events above, with the addition that the riding surface must be dry.

\subsubsection{Slope Glide Or Track Glide}
A slope glide can be done on a small hill.
Riders start on the hill, gliding down to level ground and continuing as far as they can before stopping.
This event can have a limited starting distance, or no starting distance at all, with riders gliding from a dead stop.
If it is a Track Glide, it is held on a track with the same rules as Track Coasting (see section \ref{subsubsec:track-field_alternate-optional-fun-events_coasting_track-coasting}).

\subsubsection{Downhill Glide}
A downhill race for speed.
Riders start from a standstill, or speed up to the `starting line.'
Riders are timed over a measured distance to the finish line.
Dismounts before the finish line disqualify the rider in that attempt.
Helmets are mandatory.

\subsection{Slow Balance Forward}

In Slow Balance Forward, the rider rides a distance of 10 meters in a continuous forward motion as slowly as possible without stopping, going backward, hopping or twisting more than 45 degrees to either side.
Any age group with riders of 11 years or older must use a board of 15 cm wide.
Any age group with no riders of 11 years or older must use a board of 30 cm wide at Unicon; in other conventions the host may choose to use either a 15 cm wide board or a 30 cm wide board for this age group.
Tires may overlap the edges of the board, but if the tire contacts the ground next to the board, that would be the end of that attempt.
There are no crank arm length or wheel size restrictions for this event.

Riders must wear shoes.
No other safety gear is required.

\subsubsection{Timing}
The position of the unicycle during Slow Balance is defined by the tire contact point.
In Slow Balance, the rider starts behind the starting line.
On command by the starter, the rider has 10 seconds to start forward motion and let go off the starting post.
The timer starts recording time when the tire contact point crosses the starting line.
At this moment, the rider may not be in contact with the starting post anymore.
Timers must watch the hands and the feet/wheel at the same time at that moment.
The time stops when the tire contact point crosses the finish line.

\subsubsection{Optional Penalty Rules}
At any bigger conventions where there is a large pool of judges (such as Unicon) it is recommended that the host uses a system wherein the judges may give penalties to riders who seem to make ``micro-errors'' or if the judges are in doubt whether an error was made.
Examples of micro-errors are twisting about 46 or 48 degrees, or vibrations of the wheel.
Each penalty subtracts one second from the ridden time.
Riders are still disqualified for clear errors, such as riding off the board, dismounting or twisting 90 degrees.
Using these penalty rules is especially discouraged for possible errors for which a reliable objective detection system is being used.

\subsubsection{Age Group and Final Rounds}
Age Group and Final rounds are always required.

\textbf{Age Group Round:}
\begin{itemize}
\item All riders must participate in the Age Groups.
Riders get two attempts.
\item The best 8 female and the best 8 male riders qualify for the finals.
\item For Unicon a minimum of 20 seconds is required to achieve a valid result.
For any age group with no riders of 11 years or older the minimum time is 15 seconds.
Riders who don't reach this threshold are automatically disqualified.
If your net time after penalties brings you below the minimum time, you are also disqualified.
For other competitions than Unicon, the host may adjust the threshold to a lower time or have no threshold at all.
\end{itemize}

\textbf{Final Round:}
\begin{itemize}
\item The Judging team for the Finals must consist of a single group of people that watch every rider, or (insofar available) an accurate and reliable technical means to check adherence to the rules.
\item Riders get two attempts.
\item The champion is the rider who performs the best result in the final round.
\end{itemize}

\subsection{Slow Balance Backward}
This is the same as Slow Balance Backward, with the following differences \textit {in italic}:
\begin{itemize}
\item Riders ride \textit{backward}.
\item It is an error to ride \textit{forward}.
\item For Unicon a minimum of \textit{15 seconds} is required to achieve a valid result.
For any age group with no riders of 11 years or older the minimum time is \textit{10 seconds}.
\item Any age group with riders of 11 years or older must use a board of \textit{30 cm} wide.
Any age group with no riders of 11 years or older must use a board of \textit{60 cm} wide at Unicon; in other conventions the host may choose to use either a \textit{30 cm} wide board or a \textit{60 cm} wide board for this age group.
\end{itemize}

\subsection{Slow Giraffe Race}

This is the same as slow forward, but on giraffes.
Helping hands can be used as starting posts.
No limits on size or gear ratio, but unicycles must have their pedal axle above the wheel axle, with a chain, belt, or other form of drive system.

\subsection{Stillstand}
Stillstand is a competition in which the rider attempts to balance as long as possible.
The rider cannot hop or turn the tire more than 45 degrees, and must remain on a 25 cm long, 10 cm wide, and 3 cm tall block of wood.
The competition should take place indoors on a level surface
The only required safety gear is shoes.

Each participant has 2 attempts that can be done at any time during the time window set by the host.
The host can decide to add to each of the 2 attempts a window up to 20 seconds, in which the competitor can start the number of tries needed.

The starting post is placed anywhere the participant prefers.
Time starts running when the competitor lets go of the starting post.
After time starts running, the starting post will be taken away.
Time stops at the moment when the participant rides off the board, dismounts, starts hopping or turns the tire more than 45 degrees.

There are no finals for the Stillstand competition.
The overall results will be determined by the best results for males and females respectively.

\subsection{700c Racing}

\begin{comment2016}
We believe that these rules are already included in track racing, and that this section can be removed.
\end{comment2016}


Races of any length and type can also be conducted in a 700c wheel category.
\begin{itemize}
\item Maximum bead seat diameter (BSD): 622 mm.
\item If these races are intended to exclude 24 Class wheels, sizes must be greater than 618 mm.
\item No restrictions on crank length.
\item Beyond these, 700c unicycles must comply with all other requirements for racing unicycles.
\item The host may choose age groups.
\end{itemize}

\subsection{Unlimited Track Racing}

An unlimited race is one in which there are no unicycle size restrictions.
Any size wheels, any length crank arms, giraffes or any types of unicycles (see definition in chapter \ref{chap:general_definitions}) are allowed.
All other Track racing rules apply.

\subsection{Juggling Unicycle Race}

The traditional distance is 50 meters.
Riders use the 5 meter line from the One-Foot Race, and must be juggling when they cross this line.
Three or more non-bouncing objects must be used.
If an object is dropped (hits the ground) or the juggling pattern is otherwise stopped, the rider is disqualified.
Two balls stopping in one hand during a 3-ball cascade is defined as stopping.
Riders who start by juggling four or more objects may drop one, as long as their pattern continues, unbroken, into three.
The juggling pattern must be `in control' when the rider crosses the finish line.
`Control' is determined by the Referee.

\subsection{Ultimate Wheel Race}

An ultimate wheel is a unicycle with no frame or seat.
Traditionally, for riders in age groups with a maximum age of 10 or younger the race distance is 10m, while for all other riders it is 30m.
Maximum wheel size is 618 mm (24 Class) for all ages, with 125 mm minimum crank arm length or 250 mm between pedal holes.
The host may allow other limitations, or none, if these details are announced well in advance.


\subsection{50m Fast Backward}

Riders must face and pedal backward.
The Starter lines up the rear of the tire above the start line.
Helmets are mandatory.
Timing is stopped when the rear of the tire crosses the finish line.

\subsection{Medley}

This is a race involving riding several different ways of riding.

\textbf{Example:} Forward 25m, seat in front 25m, one foot 25m, hopping 10m, with 5m transition areas.
Rules are set by the host.
Remounting is allowed.
