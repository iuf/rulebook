\chapter{Competitor Rules}

\section{Alternate, Optional or Fun Events \label{sec:track-field_alternate-optional-fun-events}}
\oldrule{2.20}
These are optional events, not guaranteed to be included in every unicycle convention.
They can be held with as much, or as little, level of formality and importance the host chooses.
Age group breakdown is also up to the host.
All of the events in this section have been run before, using these rules.
If a large convention advertises events with the names of the ones detailed in this section, they must use the rules provided here.
If hosts desire to do variations on these rules, the events must be labeled accordingly.
Example: ``Track Gliding; Modified''.
In cases such as this, hosts must remember to provide detailed rules for these events at the same time the events are announced.

\subsection{Relay (Track)}
\oldrule{2.20.1}
Usually 4 x 100m or 4 x 400m like in athletics.

The takeover zones are 20m long and must be marked on the track.
Riders may remount if necessary, and must pick up the baton if it is dropped.
The handover of the baton must be within the takeover zone.
This means that before the baton crosses the start mark of the takeover zone \textit{only} the incoming rider is in touch with the baton and at the end of the takeover zone \textit{only} the outgoing rider is in touch with the baton.
Riders may not throw the baton to make a pass and may not touch the ground with any part of their body while making a pass.
If the baton is not handed over within the marked takeover zone, the team will be disqualified.
Leaving of the lane within the takeover zone or when remounting does not result in disqualification as long as the riders do not obstruct, impede or interfere with another rider's progress.
There is no defined preparation area for the next riders as long as they stay within their lanes.

Mixed male/female teams may be used, and reasonable age groups may be used depending on the number of expected competitors of the event.
Each relay team may have any mix of ages, the age of the oldest rider determines the age group.

\subsection{Coasting Events}
\oldrule{2.20.1}
An event to determine which rider coasts the furthest distance.
Riders' coasting distances are measured from a `starting line' with a 5 meter minimum, which will be marked by a `qualifying line.'
If the rider does not cross the qualifying line it will count as a failed attempt.
The farthest distance from the line wins.
The distance is measured to the rearmost part of the rider that touches the ground when dismounting, or to the rear of the tire where the rider stops coasting.
Remounting is not allowed.
Riders must not touch any part of their tires, wheels or pedals while coasting.
Riders get two attempts.
If a rider crosses the coasting line (front of the tire) not in coasting position, he or she is disqualified in that attempt.
The riding surface should be as smooth and clean as possible, and it may be straight or curved.
Ample time must be allowed for all competitors to make some practice runs on the course before the official start.
The type of event(s) to be used should be announced well in advance of the competition.
Crank arm rules do not apply in any coasting or gliding events.

\subsubsection{Road Coasting}
\oldrule{2.20.2.1}

This event is best held on a roadway with a very slight downward slope.
Riders are allowed an unlimited distance to speed up and start coasting before the starting line.

\subsubsection{Track Coasting \label{subsubsec:track-field_alternate-optional-fun-events_coasting_track-coasting}}
\oldrule{2.20.2.2}
30 meter speed-up distance.
This event is held only on a track, or a very level, smooth surface.
Wind must be at a minimum for records to be set and broken.
This event can be compared with other races at different tracks worldwide.

\subsubsection{Downhill Coasting}
\oldrule{2.20.2.3}
This is a speed coasting event.
Riders start from a standstill, or speed up to the `starting line'.
Riders are timed over a measured distance to the finish line.
Dismounts before the finish line disqualify the rider in that attempt.
The slope must be very gradual for this event to be safe, and helmets are mandatory.

\subsubsection{Indoor Coasting}
\oldrule{2.20.2.4}
30 meter starting distance.
This event is held indoors in a gym, or on a very level, smooth surface.
Rider will coast in a circle on the outer edge of the gym, separated by cones.
Both directions are allowed for the start (clockwise or counterclockwise), and rider will have a maximum of 30m before beginning to coast.
Indoor coasting is the recommended coasting competition at a Unicon.

\subsection{Gliding Events}

\oldrule{2.20.3}
Gliding is like coasting, but with one or both feet dragging on top of the tire to provide balance from the braking action.
These events are similar to the coasting events above, with riders gliding for time or distance from a given point.
The rules are the same as for the coasting events (above) with the addition that the riding surface must be dry.
Coasting is allowed.

\subsubsection{Slope Glide Or Track Glide}
\oldrule{2.20.3.1}
A slope glide can be done on a small hill.
Riders start on the hill, gliding down to level ground and continuing as far as they can before stopping.
This event can have a limited starting distance, or no starting distance at all, with riders gliding from a dead stop.
If it is a Track Glide, it is held on a track with the same rules as Track Coasting (see section \ref{subsubsec:track-field_alternate-optional-fun-events_coasting_track-coasting}).

\subsubsection{Downhill Glide}
\oldrule{2.20.3.2}
A downhill race for speed.
Riders start from a standstill, or speed up to the `starting line.'
Riders are timed over a measured distance to the finish line.
Dismounts before the finish line disqualify the rider in that attempt.
Helmets are mandatory.

\subsection{Slow Forward}

\oldrule{2.20.4}
The object is to ride in a continuously forward motion as slowly as possible without stopping, going backward, hopping, or twisting more than 45 degrees to either side.
Two different board sizes are used: Age 0-10: 10m x 30cm.
Age 11-UP: 10m x 15cm.
The Slow Race is measured using the bottom of the unicycle wheel.
Riders start with the bottom of the wheel on the starting line.
On command by the Starter, the rider must immediately start forward motion and let go of starting posts.
The timer stops the watch when the bottom of the tire touches either the finish line, or the ground after the line on boards that end at the finish line.
Riders can be disqualified for very slight stops or backward motions, twisting more than 45º to the side, riding off the sides of the board, or dismounting.
Riders get two attempts.
There are no crank arm length or wheel size restrictions for this event.
No safety gear is required.

\subsection{Slow Backward}
\oldrule{2.20.5}
This is the same as the slow forward race \textit{except}: 0-10 ride on 60cm board, 11-Up ride on 30cm board.

\subsection{700c Racing}
\oldrule{2.20.6}

Races of any length and type can also be conducted in a 700c wheel category.
\begin{itemize}
\item Maximum wheel diameter: 768mm.
\item If these races are intended to exclude 24$"$ wheels, sizes must be greater than 618mm.
\item No restrictions on crank length.
\item Beyond these, 700c unicycles must comply with all other requirements for racing unicycles.
\item The host may choose age groups.
\end{itemize}

\subsection{Unlimited Track Racing}
\oldrule{2.20.7}

An unlimited race is one in which there are no unicycle size restrictions.
Any size wheels, any length crank arms, giraffes or any types of unicycles (see definition in chapter \ref{chap:general_definitions}) are allowed.
All other Track racing rules apply.

\subsection{Juggling Unicycle Race}
\oldrule{2.20.8}

The traditional distance is 50m.
Riders use the 5m line from the One-Foot Race, and must be juggling when they cross this line.
Three or more non-bouncing objects must be used.
If an object is dropped (hits the ground) or the juggling pattern is otherwise stopped, the rider is disqualified.
Two balls stopping in one hand during a 3-ball cascade is defined as stopping.
Riders who start by juggling four or more objects may drop one, as long as their pattern continues, unbroken, into three.
The juggling pattern must be `in control' when the rider crosses the finish line.
`Control' is determined by the Referee.

\subsection{Ultimate Wheel Race}
\oldrule{2.20.9}

An ultimate wheel is a unicycle with no frame or seat.
The traditional distance is 10m for 0-10 riders, and 30m for 11-UP riders.
Maximum wheel size is 618mm (24$"$) for all ages, with 125mm minimum crank arm length or 250mm between pedal holes.
The host may allow other limitations, or none, if these details are announced well in advance.

\subsection{50m Fast Backward}
\oldrule{2.20.10}

Riders must face and pedal backward.
The Starter lines up the rear of the tire above the start line.
Helmets are mandatory.
Timing is stopped when the rear of the tire crosses the finish line.

\subsection{Medley}
\oldrule{2.20.11}

This is a race involving riding several different ways of riding.

\textbf{Example:} Forward 25m, seat in front 25m, one foot 25m, hopping 10m, with 5m transition areas.
Rules are set by the host.
Remounting is allowed.

\subsection{Slow Giraffe Race}
\oldrule{2.20.12}

This is the same as slow forward, but on giraffes.
Helping hands can be used as starting posts.
No limits on size or gear ratio, but unicycles must have their pedal axle above the wheel axle, with a chain, belt, or other form of drive system.