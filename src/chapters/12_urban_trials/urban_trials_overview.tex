\chapter{Overview}

\begin{comment-2016}
Much of the Trials rules consist of descriptions of various things, and are not rules, as such.  This whole chapter should be reworded to make rules!
\end{comment-2016}

\section{Definition}

\begin{comment-2016}
The following text, which does not contain any rules, has more detail than the intro to other chapters.  It could be rewritten to pare it down.  It would be nice to add text saying the rider carries a score card which is signed by a judge at the completion of each section.
\end{comment-2016}

\oldrule{8.1}
The object of unicycle trials is to ride over obstacles. 
A unicycle trials competition takes place on a ``course'' containing different obstacles called ``sections''. 
Each section is worth one point, and courses typically contain 15 – 40 or more sections.

Riders earn points by successfully riding (``cleaning'') each section from start to finish. 
The objective is to earn as many points as possible by cleaning as many sections as possible.

At the end of a specified time period, the rider with the highest overall number of points (who has cleaned the most number of sections) is the winner.

\oldrule{8.2}
The competition takes place within a specified time period (2+ hours depending on the number of obstacles), on a collection of 15 to >40 independent, numbered sections of any length (typically 3 m to 20 m long). 
Sections may include narrow beams or logs, steep climbs, rocks, etc.

The average difficulty level of sections will vary between competitions depending on the ability level of the riders participating. 

At each section are posted instructions that identify the section number, its difficulty level, and a description of the section.
Section boundaries are defined by flagging tape and/or instructions that designate a start line, section boundaries, and a finish line.

\section{Rider Summary}

This section is intended as an overview of the rules, but does not substitute for the actual rules.
\begin{itemize}
\item You must wear shoes, shin guards, and a helmet.
\item There is no restriction on type of unicycle.
\item You may change unicycles during the competition.
\end{itemize}