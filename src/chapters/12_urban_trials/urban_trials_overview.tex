\chapter{Overview}

\begin{framed}
New text, not from the 2014 rulebook is in \textbf{boldface}. Text that we suggest be deleted is in \textit{italic}. Since the text has been reorganized, the original 2014 sections are marked with a boxed reference, such as \oldrule{3.4}. Meta-text, intended for the rulebook editors is boxed.
\end{framed}

\section{Definition}

\oldrule{7.1}
The object of unicycle trials is to ride over obstacles. 
A unicycle trials competition takes place on a ``course'' containing different obstacles called ``sections''. 
Each section is worth one point, and courses typically contain 15 – 40 or more sections.

Riders earn points by successfully riding (``cleaning'') each section from start to finish. 
The objective is to earn as many points as possible by cleaning as many sections as possible.

At the end of a specified time period, the rider with the highest overall number of points (who has cleaned the most number of sections) is the winner.

\section{Rider Summary}

\textbf{This section is intended as an overview of the rules, but does not
substitute for the actual rules.
\begin{itemize}
\item You must wear shoes, knee pads, shin guards, gloves, and helmet.
\item There is no restriction on type of unicycle.
\end{itemize}}