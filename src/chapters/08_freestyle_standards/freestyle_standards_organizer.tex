\chapter{Event Organizer Rules}

\oldrule{5D}
These are the guidelines by which Standard Skill competition is to be executed.
At times, however, situations may occur in which the regulations cannot be followed exactly.
This applies to minor details; not to principal rules.
For instance, if the size of the available accommodation would cause the size of the riding area to be slightly smaller than required, that can be approved by a majority vote of the judging panel.
Whatever differences from the rules are approved must be made known to all participants before competition.
Any situation that may occur for which the rules do not provide a solution, shall be solved by the Chief Judge or by a majority vote in a meeting chaired by the Chief Judge, at which all judges active in the concerned event must be present.

\section{World Champions \label{sec:freestyle_world-champions}}
\begin{framed}
We recommend that this be moved to Chapter 1 for all events.
\end{framed}

\oldrule{5.11}


Winners in the Expert category at of each event at Unicon are the \textbf{World Champions}.
In the individual events, separate titles are awarded for male and female.
Winners in the Jr. Expert category at Unicon are the \textbf{Junior World Champions}.

\section{Venue}
\section{Officials}
\section{Communication}

\subsection{Announcing Of Results}

\oldrule{5.7}
Final results will be continuously announced and/or posted for public view.
Results Sheets will be posted after each age category of an event.
The protest period begins at this point.


\section{Age Groups}

\oldrule{5.2}
\textbf{Note:} Age groups may be different for different types of event.
The minimum allowable age groups are listed for each event.
Convention hosts are free to add more age groups.
Age group is determined by the rider's age on the first day of the convention.
Junior Expert is open to all riders 0-14.
Expert is open to riders of any age, including 0-14.
Riders must state the age group in which they are entering for each artistic event in which they participate.

\textbf{Example:} Riders who enter Individual Freestyle as Experts can enter Pairs in their age group if they wish.
Riders are divided male/female in Standard Skill and Individual Freestyle, but not in Pairs or Group.

\subsection{Minimum Age Groups}

\oldrule{5.27.1}
0-14, 15-UP.
Best overall scores determine which competitors reach the Expert ranks.

\section{Practice}
\section{Race Configuration}
\section{Starting Configuration}
\section{Starting Order}
\section{Starter}
\section{False Starts \label{subsec:false_starts}}
\section{Finishes}
\subsection{Optional Race-End Cut-Off Time}
\section{Special Marathon Events}
\section{Race Distances and Distance Measurement}
\subsection {Distance Measurement for Traditional Distances}
\subsection {Distance Measurement for Other Distances}

\section{Music \label{sec:freestyle_music}}
\oldrule{5.6}
In Freestyle events, music is included in the judging and competitors should use it.
In Standard Skill music is not judged.
But background music will be provided during all Standard Skill routines, or competitors may provide their own.
Competitors may also, at their request, have no music played.
It is recommended to have one or more backup copies of all music in case of loss or damage.
For recordable disks, competitors are also recommended to test their music on multiple players to make sure it will work at competition time.

\subsection{Media Types}
\oldrule{5.6.1}
The host is required to have the capability of playing recordable CDs.
Other media types may also be supported, at the host's discretion.
The Artistic Director is responsible for announcing what media types will be supported, and making sure the necessary equipment is provided.

\subsection{Music Preparation}
\oldrule{5.6.2}
Competitors must provide their music in a type that is supported, and has been announced by the Artistic Director.
All music must be clearly labeled with the competitor name(s), age group, event type (such as Pairs), and if needed, the track number.
Whenever possible, competition music should be the first track on the CD.
The DJ (music operator) is not responsible for any errors resulting from unsupported types or mislabeled tracks.

\subsection{Music Volume}
\oldrule{5.6.3}
Volume level is controlled by the DJ, at instructions from the Chief Judge.
The base volume for Freestyle music should be loud enough to sound clear, and be heard by all.
For Standard Skill, volume level should not be loud enough to interfere with judge communication, but otherwise similar to the level for Freestyle.
Some competitors' music may start with especially loud or quiet sections, and the DJ should be advised of these so volume levels do not get compensated in the wrong direction.
Some competitors may request that their music be played at lower levels.
These requests can be made directly to the DJ.
Requests for higher volumes must be approved by the Chief Judge, who has the option of passing this responsibility to the DJ.

\subsection{Special Music Instructions}
\oldrule{5.6.4}
Some competitors may have special music instructions, such as stopping or starting the music at a visual cue, changing volume level during the performance, etc.
The DJ is not responsible for errors carrying out these instructions.
For best results, the competitor should supply a person to coach the DJ during the performance, so there are no mistakes.
If the DJ receives instructions that sound unusual, the Chief Judge should be consulted for approval.


