\chapter{2013 Freestyle \& Standard Skill \label{chap:freestyle} Overview}

\section{The Difference Between These Events}
In Standard Skill, riders demonstrate pure skill and mastery on a standard unicycle, by performing up to 18 skills they have pre-selected.
Standard Skill judging is based on the point value of the skills and quality of their execution, not the `show.' In Freestyle, riders perform to music, with costumes, props and any kinds of unicycles.
Riders are judged not only on skill, but also on how well they entertain and put on a show.
There are Individual, Pair, and Group Freestyle events.

\section{Age Groups}
\textbf{Note:} Age groups may be different for different types of event.
The minimum allowable age groups are listed for each event.
Convention hosts are free to add more age groups.
Age group is determined by the rider's age on the first day of the convention.
Junior Expert is open to all riders 0-14.
Expert is open to riders of any age, including 0-14.
Riders must state the age group in which they are entering for each artistic event in which they participate.

\textbf{Example:} Riders who enter Individual Freestyle as Experts can enter Pairs in their age group if they wish.
Riders are divided male/female in Standard Skill and Individual Freestyle, but not in Pairs or Group.

\subsection{Riders Must Be Ready}
Riders who are not ready at their scheduled performance time may or may not be allowed to perform after the last competitor in their age group.
The Chief Judge will remember to consider language barriers, and that riders may be engaged in convention work to slow them down.
Except for Standard Skill, a rider may not perform before a different set of judges than those that judged the rest of their age group.

\section{Safety Gear}
No safety gear is needed.

\section{Performance Set-Up}
Competitors are allowed a maximum of two minutes to set up their unicycles and props in the performing area.
Competitors who take too long risk being disqualified.
An extension of the set-up time can be given only by the Chief Judge and must be requested in advance.
Competitors must show a legitimate need when requesting more time, such as numerous props or complicated special effects.

\section{Interruption Of Judging}
An interruption of judging can result from material damage, injury or sudden illness of a competitor, or interference with a competitor by a person or object.
If this happens, the Chief Judge determines the amount of time left and whether any damage may be the fault of the competitor.
Re-admittance into competition must happen within the regulatory competition time.
If a routine is continued and the competitor was not at fault for the interruption, all devaluations coming forth from the interruption will be withdrawn.

\section{Music \label{sec:freestyle_music}}
In Freestyle events, music is included in the judging and competitors should use it.
In Standard Skill music is not judged.
But background music will be provided during all Standard Skill routines, or competitors may provide their own.
Competitors may also, at their request, have no music played.
It is recommended to have one or more backup copies of all music in case of loss or damage.
For recordable disks, competitors are also recommended to test their music on multiple players to make sure it will work at competition time.

\subsection{Media Types}
The host is required to have the capability of playing recordable CDs.
Other media types may also be supported, at the host's discretion.
The Artistic Director is responsible for announcing what media types will be supported, and making sure the necessary equipment is provided.

\subsection{Music Preparation}
Competitors must provide their music in a type that is supported, and has been announced by the Artistic Director.
All music must be clearly labeled with the competitor name(s), age group, event type (such as Pairs), and if needed, the track number.
Whenever possible, competition music should be the first track on the CD.
The DJ (music operator) is not responsible for any errors resulting from unsupported types or mislabeled tracks.

\subsection{Music Volume}
Volume level is controlled by the DJ, at instructions from the Chief Judge.
The base volume for Freestyle music should be loud enough to sound clear, and be heard by all.
For Standard Skill, volume level should not be loud enough to interfere with judge communication, but otherwise similar to the level for Freestyle.
Some competitors' music may start with especially loud or quiet sections, and the DJ should be advised of these so volume levels do not get compensated in the wrong direction.
Some competitors may request that their music be played at lower levels.
These requests can be made directly to the DJ.
Requests for higher volumes must be approved by the Chief Judge, who has the option of passing this responsibility to the DJ.

\subsection{Special Music Instructions}
Some competitors may have special music instructions, such as stopping or starting the music at a visual cue, changing volume level during the performance, etc.
The DJ is not responsible for errors carrying out these instructions.
For best results, the competitor should supply a person to coach the DJ during the performance, so there are no mistakes.
If the DJ receives instructions that sound unusual, the Chief Judge should be consulted for approval.

\section{Announcing Of Results}
Final results will be continuously announced and/or posted for public view.
Results Sheets will be posted after each age category of an event.
The protest period begins at this point.

\section{Protests}
Must be filed in writing, within 15 minutes from the posting of event results.
Protest against judges' scores is not permissible.
Protest is only possible against calculation mistakes or other mistakes not connected to the scoring.
The Chief Judge must resolve all protests within 30 minutes from receipt of the written form.

\section{Freestyle Judging Panel \label{sec:freestyle_judging-panel}}
There are five (or more) judges each of Technical and Presentation for Age Group competitions; five (or more) judges each of Technical and Presentation for Jr. Expert and Expert competitions (including Group).
All judges must attend a workshop provided as part of the convention schedule before the start of the Freestyle competitions.
Exceptions to workshop attendance are granted by the Chief Judge if judging rules have not changed since the previous judging experience and the judge has high Accuracy Scores.
Unless otherwise noted, judges at a Unicon must either speak English or have translation assistance for the specified language while judging.
Judges at other unicycle conventions should speak the dominant language of that convention or have translation assistance.

Judges' names must be provided to the Chief Judge (via email, FAX, or postal mail) by at least one month prior to the start of the unicycle convention and include the number of freestyle conventions where they have been a competitor, judge, or simply in the audience.
See section \ref{subsec:freestyle_judging-panel_nominating-freestyle-judges} for description of which teams/countries are required to provide judges.
Judges must be at least 15 years of age at the start of the event.
Judges are recommended to be a current freestyle competitor, a former freestyle competitor, an active coach of freestyle routines, a proven judge at prior competitions, or an avid spectator who has observed many freestyle routines.
Details about the Standard Skill judging panel are covered in section \ref{sec:freestyle_std-judging-panel}.

\subsection{Selecting Judges \label{subsec:freestyle_judging-panel_selecting-judges}}
A person should not judge an event if he or she is:
\begin{itemize}
\item A parent, child or sibling of a rider competing in the event.
\item An individual or team coach, manager, trainer, colleague who is member of the same club specified in the registration form, colleague's family etc.
of a rider competing in the event.
\item More than one judge from the same family judging the same event at the same time.
\end{itemize}
If the judging pool is too limited by the above criteria, restrictions can be eliminated starting from the bottom of the list and working upward as necessary only until enough judges are available.
If there are some candidates who have the same level of restrictions and judging score, their agreement about publishing the results need to be considered.
The eliminations must be agreed upon by the Chief Judge and Artistic Director, or next-highest ranking artistic official if the Chief Judge and Artistic Director are the same person.

\subsection{Assignment Of Age Group Judges}
Judges will be chosen from the list of judges as provided in section \ref{subsec:freestyle_judging-panel_nominating-freestyle-judges}.
Judges who are competing in the event just before or just after the current category are strongly suggested to be eliminated from the list.
Judges will also be eliminated from the list for the current category as described in section \ref{subsec:freestyle_judging-panel_rating-judge-performance}.
The final selection of judges will be chosen based on their accuracy scores from the remaining list.

\subsection{Assignment Of Expert (And Junior Expert) Judges \label{subsec:freestyle_judging-panel_assignment-of-expert-judges}}
Assignments for Expert and Jr. Expert judges will be made by the Chief Judge using the most qualified of all judges available.
Qualifications are determined in the following order of importance: 
\begin{itemize}
\item Highest judging accuracy scores obtained while judging age group (age groups judges must have a minimum of five entrants) or other Jr. Expert and Expert events.
\item Greatest amount of Jr. Expert and Expert judging experience.
\item Greatest amount of international judging experience.
\item Greatest number of Freestyle competition experienced (viewed, judged, or as a competitor).
\end{itemize}
Judges who are competing in the event just before or just after the current category are eliminated from the list.
Judges will also be eliminated from the list for the current category as described in section \ref{subsec:freestyle_judging-panel_selecting-judges}.
Judges will also be eliminated from the list if they exhibit Judging weaknesses during their Age Group judging as described in Section \ref{subsec:freestyle_judging-panel_rating-judge-performance}.
At Unicons, if more than five judges each of Technical and Presentation remain, judges who have not judged at a previous Unicon will be removed from the list.
If there are still more than five each then the final list of judges for the category will be chosen by accuracy scores as defined in section \ref{subsec:freestyle_judging-panel_calculating-accuracy-scores}.

\subsection{Judging Panel May Not Change}
The individual members of the judging panel must remain the same for entire age groups; for example one judge may not be replaced by another except between age groups.
In the event of a medical or other emergency, this rule can be waived by the Chief Judge.

\subsection{Rating Judge Performance  \label{subsec:freestyle_judging-panel_rating-judge-performance}}
Judges are rated by comparing their scores to those of other judges at previous competitions.
Characteristics of Judging Weaknesses
\begin{itemize}
\item \textbf{Excessive Ties:} A judge should be able to differentiate between competitors.
Though tying is most definitely acceptable, excessive use of tying defeats the purpose of judging.
\item \textbf{Group Bias:} If a judge places members of a certain group or nation significantly different from the other judges.
This includes a judge placing members significantly higher or significantly lower (a judge may be harsher on his or her own group members) than the other judges.
\item \textbf{Inconsistent Placing:} If a judge places a large number of riders significantly different from the average of the other judges.
\end{itemize}

\subsection{Re-Instating Judges}
If a judge has been labeled as having a Judging Weakness, they may have a chance to be re-instated on the list by:
\begin{itemize} 
\item Discuss with the Chief Judge the scores that were Tied, Biased, or Inconsistent.
\item Practice judge on at least two categories with at least 4 competitors.
\end{itemize}
If the practice judging shows no further examples of Judging Weakness, they may be reinstated on approval by the Chief Judge and Artistic Director.
If the Chief Judge and Artistic Director are the same person, then the next highest-ranking official must agree to the reinstatement.

\subsection{Calculating Accuracy Scores \label{subsec:freestyle_judging-panel_calculating-accuracy-scores}}
The Chief judge should decide to replace a judge if he/she shows signs of weakness.
To find the right decision, the chief judge may use heuristics, statistical analytics, etc. as indicators.

\subsection{Nominating Freestyle Judges \label{subsec:freestyle_judging-panel_nominating-freestyle-judges}}
Parties (Countries/Clubs) that participate at competitions must nominate judges in relation to the number of Freestyle participants they send (see table below). 
After registration finishes, the chief judge will send a request to all parties.
The request contains the compiled number of minimum judges per party and a question to nominate the candidates.
Judge nominations include the experience of the judges (such as previous competitions, how long he/she has been judging, agegroup/expert judging or other relevant qualifications such as educations).

\begin{tabular}{|l|l|}
\hline
\textbf{Number of Participants per Party} & \textbf{Minimum Number of Judges per Party} \\
\hline
$<$5 & 0 \\
\hline
5-10 & 2 \\
\hline
11-20 & 3 \\
\hline
21-30 & 4 \\
\hline
$>$30 & 5 \\
\hline
\end{tabular}

\subsection{Not Providing Judges}
At Competitions, parties that are unable to provide their required number of judges (either Group or Individual/Pairs) may have their competitors removed from that competition.
Exceptions will be granted on a special basis with a letter to the Chief Judge, Artistic Director, and Convention Director. 
\textbf{Note:} A party that isn’t able to nominate their minimum judges can ask a party that has more than the required number of minimum judges to nominate their additional judges as their own.

\subsection{Judges Workshop}
The hosts of the convention must provide for a judge's workshop at least 24 hours prior to the start of the Freestyle competition.
A minimum of 3 hours must be set aside, in a classroom or similar environment.
If possible, it is strongly recommended to have more than one workshop to accommodate schedules.
Variations on this can be approved by the Chief Judge.
Workshop schedule(s) must be announced to all judges at least three weeks prior to the start of the competition.

Judges should have read the rules prior to the start of the workshop.
The workshop will include a practice judging session.
Each judge will be required to sign a statement indicating they have read the rules, attended the workshop, agree to follow the rules, and will accept being removed from the list of available judges if their judging accuracy scores show Judging Weaknesses.

\section{Scoring}
In all events except Standard Skill, the scores of each judge are transferred into placing points, which represent the ranking of each competitor by that judge.
The highest scoring competitor gets 1 placing point, the next one gets 2, and so on.

\textbf{Note:} The ranking number, or highest placing point available for a competitor depends on the number of entries in that category.
If two or more competitors have the same score, they are awarded equal portions of the total number of placing points available for the places they occupy in the ranking.

\textbf{Example:} Seven competitors.
Four of them tie for 2nd place.
7th place gets 7 points, 6th place gets 6 points, and 1st place gets 1 point.
For the other four competitors, add up the other placing points numbers: $2+3+4+5=14$.
Divide this by the number of competitors (4) to get 3.5 placing points each.

\subsection{Removing The High And Low}
After determining placing points as above, discard the highest and lowest placing score for each rider.
If Rider A has scores of 1,2,1,3,2, take out one of the ones, and the three.
Then Rider A has 1,2,2, for a total of 5.
If Rider B has scores of 2,2,2,2,2, he will end up with 2,2,2, a total of 6.
The winner is the competitor with the lowest total placing points score after the high and low have been removed.

\subsection{Ties}
If more than one competitor has the same placing score after the above process, those riders will be ranked based on their placing scores for Technical.
The scoring process must be repeated using only the Technical scores for the tied riders to determine this rank.
High and low placing scores are again removed in the process.
If competitors' Technical ranking comes out equal, all competitors with the same score are awarded the same place.

\section{World Champions \label{sec:freestyle_world-champions}}
Winners in the Expert category at of each event at Unicon are the \textbf{World Champions}.
In the individual events, separate titles are awarded for male and female.
Winners in the Jr. Expert category at Unicon are the \textbf{Junior World Champions}.

