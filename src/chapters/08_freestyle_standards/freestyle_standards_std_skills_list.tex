\chapter{List of Standard Skills \label{chap:freestyle_std-skills-list}}

\section{General Remarks About Standard Skill Riding  }
Only figures listed in the following skills list can be used for the assembly of Standard Skill routines.

\subsection{Riding Position}
Unless stated differently in a figure description, it is to be executed with the rider seated and with both feet on the pedals.

\subsection{Body Form}
The rider must show proper body form and shall not change this form during the execution of the entire figure.
Proper body form must also be shown for the figure before and after transitions, even if not listed on the judging sheet.
The body form may be relaxed when not performing figures, except for figures before and after transitions.

\subsection{Riding Direction}
Unless stated differently, all riding figures are to be performed riding forward, this being the direction in which the rider faces.

\subsection{Pattern}
Unless stated differently in a figure description, it is to be executed in a line.
Exceptions are mounts, stationary skills and transitions, axis skills, single and counted short skills, which can be executed at any spot in the riding area.

\subsection{Transitions, And Single Short Skills}
Unless stated differently in the description of a transition, it starts and ends with the rider seated with both feet on the pedals.
Before and after transitions, and single short skills entered on the score sheet as figures, at least one revolution of the wheel must be ridden in the start and end positions.
If the start or end position of a transition or single short skill is a counted short skill, that skill must be executed at least 50\% as described, whether or not it is listed on the judging sheet.

\textbf{Example 1:} For the transition ``Riding to Seat in Front'', the rider must ride at least one full revolution of the wheel with the seat in front.

\textbf{Example 2:} For the single short skill, 180$^\circ$ uni spin to idling 1ft, the rider must idle one foot $2 \frac{1}{2}$ cycles.

\subsection{Axis Skills}
Unless stated differently in the description, it starts and ends with the rider seated with both feet on the pedals.
Before axis skills entered on the score sheet as figures, at least one revolution of the wheel must be ridden in the start position.
After axis skills, at least one revolution of the wheel must be ridden.
The ending position is not required to be the same as the starting position.

\subsection{Mounts}
Unless stated differently in the description of a mount, it is to end with the rider seated with both feet on the pedals.
After all mounts listed on the judging sheet as figures, at least one full revolution of the wheel must be ridden in the end position.
For mounts ending in counted short skills, the skill must be executed at least 50\% as described, whether or not it is listed on the judging sheet.

\textbf{Example:} For the Side Mount, the rider must ride at least one full revolution of the wheel in the riding position after mounting.

\subsection{Seat Out Figures}
Unless stated differently in seat out figures, the rider shall have no contact with the seat other than one hand holding the seat.
The hand holding the seat as well as the corresponding arm shall be extended away from the rider's body and shall not touch any part of the rider's body.

\subsection{One Foot Figures}
Unless stated differently in one foot figures, the free foot is to be placed on the frame so that there is no contact between the free foot and any rotating part of the unicycle.

\subsection{Wheel Walk Figures}
Unless stated differently in wheel walk figures, the feet are to push only the tire, and shall have no contact with the pedals or crank arms.

\subsection{Coasting}
Unless stated differently in coasting figures, the feet are to have no contact with any rotating part of the unicycle (pedals, crank arms, or tire).

\subsection{Transitions, Single Short Skills, Mounts Involving Seat Out Skills}
Unless stated differently in the description of the figure, those beginning or ending in seat out skills are allowed to have one or both hands touching the seat, and the seat touching the body for the final or first hop, idle, or revolution.
This includes, but is not limited to: unispins to seat out skills, transitions into and out of seat in front or back, leg around skills, side ride to seat in front, transitions out of seat drag in front or back, hopping wheel to pedals.

\subsection{Transitions To/From Stand Up Wheel Walk}
In all transition skills from/to stand up wheel walk position, a second foot may briefly touch the wheel during the transition, but only one foot pushes the wheel forward.
Unless clearly stated in the description, the rider must perform stand up wheel walk forward.

\subsection{Spins And Pirouettes}
The rider must make a minimum of three full rotations for spins and pirouettes.
Spins must be ridden around a fixed point and must not exceed a 1 meter diameter.
If rider rotates more than required minimum number, the last required rotations are judged for spins.
Pirouettes must be executed on 1 spot and the pedals may not move backward or forward during the pirouette.
If rider rotates more than required minimum number, the first required rotations are judged for pirouettes.

\subsection{Leg Around Skills}
All variations may begin or end with feet on the cranks or pedals and begin or end either riding, idling, or hopping unless otherwise specified.

\subsection{Idling Figures}
In idling figures, a minimum of 5 consecutive cycles (back and forth motions) must be executed.

\subsection{Twisting Figures}
In twisting figures, a minimum of 5 consecutive cycles (side to side motions) must be executed.

\subsection{Stillstands}
The minimum time for stillstands is 3 seconds.

\subsection{Hopping Figures}
In hopping figures, a minimum of 5 consecutive hops must be executed.
