\chapter{Overview \label{chap:road_racing}}

\section{Definition}

%% remember to include other distance races.

\textbf{Road races are longer distance races held on paved roads or paved bike
paths. These rules specifically apply to 100k, Marathon, and 10k races, but
may also be applied to other road races, such as a Time Trial or a
Criterium.}

\oldrule{4.1}
\textit{These are races held usually on roadways or bike paths, generally for longer distances than our events on the track.
All riders may race together and be separated by age group afterward}

\oldrule{4.5}
\textit{Traditional road race distances have been:
\begin{itemize}
\item 10k with Unlimited and Standard 24$"$ classes, and
\item Full Marathon (42.195k) with Unlimited and possibly Standard 700c classes.
\end{itemize}
However, any distances or wheel size classes can be used for Road Races.}

\section{Rider Summary}

\textbf{This section is intended as an overview of the rules, but does not
substitute for the actual rules.
\begin{itemize}
\item You must wear shoes, gloves, and helmet.
\item Personal music systems are not allowed for any races on public roads where there may be motorized traffic.
\item Water and food stations are at the discretion of the host.
\item Road racing events have wheel size, crank length, and gearing 
requirements that you need to be aware of.
\item Road races are run in two classes: standard and unlimited.
\item Be aware of the rules regarding false starts, passing, dismounts, and illegal riding, repair or replacement of a unicycle, and protests.
\item There may be a race cut-off time, as communicated by the host.
\end{itemize}}
