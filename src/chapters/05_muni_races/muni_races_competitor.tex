\chapter{Competitor Rules}

\section{Safety}
\oldrule{4.3}
Riders must wear shoes, kneepads, gloves and a helmet (see definitions in chapter \ref{chap:general_definitions}).

\textit{The IUF allows no exceptions to this for muni events. Additional equipment such as shin, elbow or ankle protection are optional.}

\textbf{Water and food are the responsibility of the rider. Hosts may offer food and water stations at their discretion.}

\section{Unicycles}

\oldrule{4.1}
There are no restrictions on wheel size, crank arm length, brakes, or gearing.

\section{Rider Identification}

\begin{framed}
Something probably needs to be said about rider number or the wearing of a chip.  This seems to be missing from the existing rules.
\end{framed}

\section{Event Flow: Uphill Race \label{sec:muni_uphill}}

\begin{framed}
Note: It would be clearer to remove the text about the no dismount variant, and call it a special race.
\end{framed}

\oldrule{4.9}
An Uphill muni race challenges a riders ability to climb.
Courses may be short and steep or longer, endurance-related challenges. 
Generally it is a timed event, but on an extremely difficult course, riders can be measured as to how far they ride before dismounting.
The race can be offered as a no-dismounts challenge, which either measures who gets the farthest, or disqualifies anyone who doesn't complete the distance without a dismount.
Multiple tries can be allowed, or the race can be a simple timed event.

\subsection{Starting}

\textbf{The starting mode for the uphill race is usually individual 
start, although the host has the discretion to choose the starting 
mode. Riders start with the fronts of their tires (forwardmost part of wheel) behind the nearest edge of the starting line.}

\begin{framed}
Can they have rolling starts?
\end{framed}

\subsection{False Starts}

\begin{framed}
Are there any rules for false starts?
\end{framed}

\subsection{Passing}

\begin{framed}
We moved the rule here, rather than pointing to it.
\end{framed}

No physical contact between riders is allowed.
Riders must maintain a minimum of one (24$"$) wheel diameter (618mm as judged by eye) between each other when passing, and at all other times.
This is measured from wheel to wheel, so that one rider passing another may come quite close, as long as their wheels remain at least 618mm apart.

\subsection{Dismounts}
\oldrule{4.8}
Dismounts are allowed in all muni races unless otherwise noted.
In mass-start events, dismounted riders must yield to mounted riders behind them as quickly as possible after a dismount, and until re-mounted.
Riders may not impede the progress of mounted riders when trying to mount.
If necessary they must move to a different location so mounted riders can pass.
If riders choose not to ride difficult sections of the course, they must not pass any mounted riders while walking or running through them.
In time trial-type events, see below for variations based on the other event details.
Violations of these non-riding rules may result in disqualification or a time penalty, to be determined and announced before the race start.
Riders must also ride completely across the finish line, as described in section \ref{sec:track-field_finishes}.

\oldrule{4.9.1}
If the Uphill race is run as a time trial, riders are intended to ride the entire distance.
In the event of a dismount, the rider must remount the unicycle:
\begin{enumerate}[(a)]
\item At the point where the dismount occurred if the unicycle falls back down the course toward the start.
\item Where the unicycle and/or rider come to a stop after dismounting.
Excessive running/walking/stumbling after a dismount may be grounds for a
penalty at the discretion of the Referee.
\item Riders may also choose to back up (toward the start line) from one of those spots to remount, if they prefer the terrain there.
\end{enumerate}


\subsection{Illegal Riding}

\begin{framed}
What needs to go here?
\end{framed}

\subsection{Finishes}

\begin{framed}
What needs to go here?
\end{framed}

\section{Event Flow: Downhill Race \label{sec:muni_downhill}}

\oldrule{4.10}
A Downhill muni race is a test of speed and ability to handle terrain. The downhill race is a timed race. A typical distance is 2.5 km.

\subsection{Starting}

\textbf{The starting mode for the downhill race is usually individual 
start, although the host has the discretion to choose the starting 
mode. Riders start with the fronts of their tires (forwardmost part of wheel) behind the nearest edge of the starting line.}

\begin{framed}
Can they have rolling starts?
\end{framed}

\subsection{False Starts}

\begin{framed}
Are there any rules for false starts?
\end{framed}

\subsection{Passing}

\oldrule{4.4}
Unless otherwise noted, non-lane passing rules apply (see section \ref{subsec:track-field_lane-use_non-lane-races}).

\subsection{Dismounts}

\oldrule{4.10.1}
Dismounted riders must not impede the progress of, or pass mounted riders.
They must remain aware of riders coming from behind, and not block them with their
unicycles or bodies.
Running is not allowed, except momentarily to slow down after a dismount.
Riders may walk if necessary.
Riders may receive a time penalty or be disqualified if they disregard this rule.

\subsection{Illegal Riding}

\begin{framed}
What needs to go here?
\end{framed}

\subsection{Finishes}

\begin{framed}
What needs to go here?
\end{framed}

\section{Event Flow: Cross Country (XC) Race\label{sec:muni_xc}}

\oldrule{4.11}
\textbf{The Cross Country race is an off-road distance rance that challenges riders' fitness and ability to ride fast on rough terrain. A typical distance is 10km.}

\subsection{Starting}

\textbf{The starting mode for the Cross Country race is usually individual 
start, although the host has the discretion to choose the starting 
mode. Riders start with the fronts of their tires (forwardmost part of wheel) behind the nearest edge of the starting line.}

\begin{framed}
Can they have rolling starts?
\end{framed}

\subsection{False Starts}

\begin{framed}
Are there any rules for false starts?
\end{framed}

\subsection{Passing}

Unless otherwise noted, non-lane passing rules apply (see section \ref{subsec:track-field_lane-use_non-lane-races}).

\subsection{Dismounts}

\oldrule{4.11.1}
If the event is held as a time trial, dismounted rider restrictions must be announced before the start of the race.
Depending on course length and difficulty, dismounted riders may be required to walk, or walk only limited distance, or have no restrictions at all.

\subsection{Illegal Riding}

\begin{framed}
What needs to go here?
\end{framed}

\subsection{Finishes}

\begin{framed}
What needs to go here?
\end{framed}

