\chapter{Event Organizer Rules}

\section{Venue}

X-Style should be held in a gym suitable for Freestyle riding.%comment2016 better text next time

\subsection{Size Of Performing Area}

The minimum size of the performing area for an X-Style event at an international championship or Unicon is 28 x 15 meters.

The minimum size for all other X-Style events must be at least 12 x 9 meters.
If the performing area is smaller than 28 x 15 meters, hosts must publicize the dimensions of the available performing area as far in advance of the competition as possible, but at least one month prior.

\section{Officials}

The host must designate the following officials for X-Style:
\begin{itemize}
\item X-Style Director
\item Chief Judge
\end{itemize}

\section{Communication}

\subsection{Publishing Results}
The published results contain the riders total ranking in order with their summed placing points and the anonomized results of the judges with their ranking for each rider.

The event director can choose the length of the runs, but it must be announced at least one month before the competition if it differs from the recommended format below.
Runs must be between 1 and 2 minutes 30 seconds.

\section{Judges Workshop}

The host must schedule a judges workshop for training X-Style judges.
This may be held just before the competition if the riders will also be judges.

\section{Age Groups and Categories}

There can be two distinct tournaments for junior (age 14 or younger) and senior (age 15 and older) riders.

The host can decide to order the riders by age and then split them into starting groups.
The host is also allowed to hand out awards to intermediate winners.
This can be motivating for younger riders.

\begin{comment2016}
\section{Music}

What are the requirements on the host regarding the sound system?
\end{comment2016}
