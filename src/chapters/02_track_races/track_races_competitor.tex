\chapter{Competitor Rules}

\section{Safety}

\oldrule{2.3}
Riders must wear shoes, knee pads and gloves (definitions in chapter \ref{chap:general_definitions}).

\oldrule{2.3}
Riders on wheels larger than 24$"$ (or with gearing) must also wear helmets.

\section{Unicycles}

\oldrule{2.2}
Only standard unicycles may be used. %comment-2016 We need a better name than standard
Riders may use different unicycles for different racing events, as long as all comply with the rules for events in which they are entered.

\oldrule{2.2.1}
For events divided by wheel size, there is a maximum allowable tire diameter and minimum crank arm length for each category:

\begin{longtable}{|p{3cm}|p{5cm}|p{2cm}|p{2cm}|}
\hline
\textbf{Unicycle} & \textbf{Maximum wheel size} & \textbf{Minimum crank arm length} & \textbf{Gearing allowed?}\\
\hline
Standard 700c & rim bead seat diamter (BSD) 
no larger than 622mm & no limit & no \\
\hline
Standard 24$''$ & tire no larger than 618mm & 125mm & no \\
\hline
Standard 20$''$ & tire no larger than 518mm & 100mm & no \\
\hline
Standard 16$''$ & tire no larger than 418mm & 89mm & no \\
\hline
\end{longtable}

\oldrule{1.2.2}For any tire in question, its outside diameter must be accurately measured.

\oldrule{2.2.2}
Crank arm length is measured from the center of the wheel axle to the center of the pedal axle. Longer sizes may be used.

\oldrule{2.2.2}
In all track racing events, shoes must not be fixed to the pedals in any way (no click-in pedals, toe clips, tape, magnets or similar).%comment-2016 what about track unlimited races?

\section{Rider Identification}

Riders must wear their race number clearly visible on their chest so that it is visible during the race and as the rider crosses the finish line.
Additionally, the rider may be required to wear a chip for electronic timing.

\section{Protests}

Protests must be filed on an official form within 30 minutes of the posting of event results.
For a large event (like Unicon) this period should be extended to 60 minutes, if possible.
The time may also be extended for riders who have to be in other races during the protest period.
All protests will be handled within 30 minutes from the time they are received.
Mistakes in paperwork, inaccuracies in placing, and interference from other riders or other sources are all grounds for protests.
All Referee decisions are final, and cannot be protested.

\section{Wheel Size Categories}

\begin{comment-2016}
This paragraph is pretty awkward, and could easily be improved.
\end{comment-2016}

\oldrule{2.1.3}
Wheel sizes for track racing are 20$"$, 24$"$ and 700c.
Additional groups for 16$"$ or other wheels can be added.
When not otherwise specified, 24$"$ is the maximum wheel size above age 10.
For age groups with a maximum age of 10 or younger, the maximum wheel size is 20$"$ (or less, if smaller sizes are also used).
The youngest age group for 24$"$ wheels should have a minimum age of 0, so riders 10 and younger have the option of racing on 24$"$ with those groups (e.g. 0-13 or 14-16).
All riders in age groups with a maximum age of 10 or younger will race a 10m Wheel Walk, and 10m Ultimate Wheel, if used (instead of 30m).

\section{Event Flow}

\oldrule{2.19}
These races should be part of every Unicon:

\subsection{100m Race}

In the 100m race, riders must stay in their lane, and a dismount results in disqualification.

\subsection{400m Race}

In the 400m race, riders must stay in their lane, and a dismount results in disqualification.

\subsection{800m Race}

In the 800m race, riders start in a lane, but at some point (usually the first turn) non-lane racing rules apply.
Dismounts are allowed.

\subsection{One Foot Race}
\oldrule{2.19.1}
Riders may pedal with both feet for the first 5 meters, but must be pedaling with only one foot after crossing the 5m line.
The 5m line is judged by looking at the tire contact point.
This means that the foot must have left the pedal when the unicycle tire is touching the 5m line on the track.
The non-pedaling foot may or may not be braced against the unicycle fork.

\subsection{Wheel Walk Race}

\oldrule{2.19.2}
Riders start mounted, with their feet on the tire, and propel the unicycle only by pushing the tire with their feet.
No contact with pedals or crank arms is allowed.
No crank arm restrictions.

\subsection{Riders Must Be Ready}

\oldrule{2.4.1}
Riders must be ready when called for their races.
Riders not at the start line when their race begins may lose their chance to participate.
The Starter will decide when to stop waiting, remembering to consider language barriers, and the fact that some riders may be slow because they are helping run the convention.

\subsection{Starting}

\oldrule{2.4}
Riders start mounted, holding onto a starting post or other support.
Unicycle riders need to be leaning forward before the starting gun fires, so the Starter will give a four-count start.
Example: ``One, two, three, BANG!''
This allows riders to predict the timing of the gun, for a fair start.

As an alternative a start-beep apparatus can be used.
In that case we have a six-count start.
Example: ``beep - beep -beep - beep - beep - buup!''
The timing between beeps is one second.
The first 5 beeps have all the same frequency.
The final tone (buup) has a slightly higher frequency, so that the racer can easily distinguish this tone from the rest.

Riders start with the fronts of their tires (forward most part of wheel) behind the edge of the starting line that is farthest from the finish line.
Rolling starts are not permitted in any race.
However, riders may start from behind the starting line if they wish, provided all other starting rules are followed.
Riders may lean before the gun fires, but their wheels may not move forward at any time.
Rolling back is allowed, but nothing forward.
Riders may place starting posts in the location most comfortable for them, as long as it doesn't interfere with other riders.

\subsection{False Starts}

\oldrule{2.5}
A false start occurs if a rider's wheel moves forward before the start signal, or if one or more riders are forced to dismount due to interference from another rider or other source. 

\subsection{Lane Use}

\begin{comment-2016}
This could be simplified a bit.  Basically, there are lane races, non-lane races, and races that transition from the former to the latter.
\end{comment-2016}

\oldrule{2.8}
In most races, a rider must stay in his or her own lane, except when the rider has to swerve to avoid being involved in a crash.
In all other cases, a rider who goes outside their lane is disqualified.
Going outside a track lane means that the tire of the unicycle touches the ground outside his assigned lane.
Riding on the marking is allowed.
No physical contact between riders is allowed during racing.
The 400m race is started with a stagger start.
The 800m race may be started in one of two ways:
\begin{itemize}
\item \textbf{Waterfall Start:} This is a curved starting line that places all riders an equal distance from the first turn.
If a waterfall start is used, non-lane rules apply (see below).
\item \textbf{Stagger Start:} Riders are started in separate lanes, at separate locations.
They must stay in their lanes for a specified distance before they may `cut in' to the inside lanes.
Lane rules apply only up to this point.
\end{itemize}

\subsection{Passing in Non-Lane Races \label{subsec:track-field_lane-use_non-lane-races}}

\oldrule{2.8.1}
This applies to 800m and other events without lanes.
No physical contact between riders is allowed.
In track races, an overtaking rider must pass on the outside, unless there is enough room to safely pass on the inside.
Riders passing on the inside are responsible for any fouls that may take place as a result.
Riders must maintain a minimum of one (24$"$) wheel diameter (618 mm as judged by eye) between each other when passing, and at all other times.
This is measured from wheel to wheel, so that one rider passing another may come quite close, as long as their wheels remain at least 618 mm apart.
The slower rider must maintain a reasonably straight course, and not interfere with the faster rider.

\oldrule{2.11}

\subsection{Dismounts}

\oldrule{2.12}
A dismount is any time a rider's foot or other body part touches the ground.
Except for the 800m, Relay races, and other races where this is announced in advance, if a rider dismounts, he or she is disqualified.
In races where riders are allowed to remount and continue, riders must immediately remount at the point where the unicycle comes to rest, without running.
If a dismount puts the rider past the finish line, the rider must back up and ride across the line in control, in the normal direction.

\subsection{Assisting Racers}

\oldrule{2.13}
In races where riders are allowed to remount, the riders must mount the unicycle completely unassisted.
Spectators or helpers may help the rider to his or her feet and/or retrieve the dropped unicycle, but the rider (and the unicycle) may not have any physical contact with any outside object or person, including a starting block under the wheel, when mounting.

\subsection{Illegal Riding}

\oldrule{2.14}
This includes intentionally interfering in any way with another rider, deliberately crossing in front of another rider to prevent him or her from moving on, deliberately blocking another rider from passing, or distracting another rider with the intention of causing a dismount.
A rider who is forced to dismount due to interference by another rider may file a protest immediately at the end of the race.
Riders who intentionally interfere with other riders may receive from the Referee a warning, a loss of placement (given the next lower finishing place), disqualification from that race/event, or suspension from all races.

\subsection{Second Attempt After Interference}
\oldrule{2.7}
If a rider is hindered due to the actions of another rider, or outside interference, either during the start or during the race, he may request to make a second attempt.
The Referee decides if the request is granted. In non-lane races, if a rider is forced to dismount due to a fall by the rider immediately in front, it is considered part of the race---not a reason to grant a second attempt---and both riders may remount and continue.
The Referee can override this rule if intentional interference is observed.

If the request is granted, it may occur that the rider has to ride his second attempt with another age group.
If all heats are finished, the rider decides if he wants company or not.
He can pick the riders, but cannot hold up the proceedings to wait for them, if other riders are available.
The resulting time of the accompanying riders is not official.
The Referee has the final say as to which extra riders are allowed to participate in such heats.

A second attempt must not be granted in the case where a rider is disqualified based on something that happened before he was hindered.

If the rider is allowed to do a second attempt and decides to do so, the first run is canceled and only the second run counts regardless of the result.
In the case where a second attempt was incorrectly granted, for example when the rider was disqualified based on something that happened before he was hindered, the result of the second attempt does not count and the result from the first run stands.

\subsection{Finishes \label{sec:track-field_finishes}}

\oldrule{2.6}
These are determined by the front of the tire crossing over the edge of the finish line that is nearest to the starting line.
Riders are timed by their wheels, not by outstretched bodies.
Riders must cross the line mounted and in control of the unicycle.
``Control'' is defined by the rearmost part of the wheel crossing completely over the finish line with the rider having: 
\begin{enumerate}
\item[(a)] Both feet on the pedals in normal races; or 
\item[(b)] One foot on a pedal in one foot races; or 
\item[(c)] At least one foot on the wheel in wheel walk races.
\end{enumerate}
In races where dismounting is allowed (800m, Relay, etc.), in the event of a dismount at the finish line the rider must back up, remount and ride across the finish line again.
In races where dismounting is not allowed, the rider is disqualified.

\section{Finals}

\begin{comment-2016}
This section is wordy, and should be rewritten.  Some text belongs elsewhere.
\end{comment-2016}

\oldrule{2.1.4}
At Unicons, a `final' must be held for each of the following races: 100m, 400m, 800m, One Foot, Wheel Walk, and IUF Slalom. 
For any other Track \& Field discipline, a `final' may be held at the discretion of the organizer, after all age group competition for that discipline has been completed.

For disciplines that are run in heats, such as 100m races or relay races, this will take the form of a final heat. 
For disciplines that are not run in heats, such as IUF slalom or slow race, the final will take the form of successive attempts by the finalists.

The riders posting the best results regardless of age in the age group heats are entitled to compete in the final.
They can be called ``finalists''.
For each final, the number of finalists (finalist teams in case of relay) will be eight, unless for an event that uses lanes, the number of usable lanes is less than eight.
In that case the number of finalists equals the number of usable lanes.
Finals are composed regardless of age group, but male and female competitors are in separate finals.

Finals are subject to the same rules as age group competition, including false start rules and number of attempts.

The best result in a final determines the male or female Champion for that discipline (World Champion in the case of Unicon).

If a finalist disqualifies, gets a worse result, or doesn’t compete in the final, his/her result in age group competition will still stand.
The male and female winners of the finals will be considered the Champions for those disciplines, even if a different rider posted a better result in age group competition.
Speed records can be set in both age group competition and finals.

In disciplines for which no finals are held, finalist status will still be awarded on the basis of results in age group competition.
Accordingly, riders posting the best results in each discipline are the Champions for that discipline.

\section{IUF Slalom}

\oldrule{2.19.3}
\begin{figure}[h]
\begin{center}
\includegraphics{iuf_slalom}
\end{center}
\vspace{-20pt}
\caption{IUF Slalom Course \label{fig:iuf_slalom}}
\vspace{-10pt}
\end{figure}
Pictured here is the IUF Slalom, in which you must ride around 10 cones in the correct pattern.
Arrows marked on the ground should indicate the direction of the turns for riders unfamiliar with the course.
The rider has to start directly behind the Start line.
The Starter gives the opening, and then the competitor has to start during the next 3 seconds.
The timer is started when any defined point of the tire (for example the part that crosses a low light beam) crosses the start line, and stops when a similar point of the tire crosses the finish line.
If the rider has not yet started after 3 seconds, the timer will start counting anyway.
The rider is not disqualified for this.
Time measurement at start and finish line must be identical to insure accurate time measurement.
It must be secured that riders do not gain momentum before crossing the start line (no flying starts).
Remounting is not allowed. 
Cones may be hit, but not knocked over.
vThe course must be followed correctly, including the direction of turns.
The last cone must be completely circled before the rider's time is taken at the finish line.
Riders who go the wrong way around a cone can go back and make the turn the correct way with the clock still running.
The cones used are plastic traffic cones.
For official competition, cones must be between 45 and 60 cm tall, with bases no more than 30 cm square.
The course must be set up accurately.
The proper positions of the cones should be marked on the ground for a cone to be replaced quickly after it has been knocked over.
Riders get two attempts.
