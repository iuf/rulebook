\chapter{Judges and Officials Rules}

\oldrule{2.3}
The Referee has final say on whether a rider’s safety equipment is sufficient. The Starter will remove from the starting line-up any riders not properly equipped to race, including riders with dangerously loose shoelaces. 

\section{Starter}

\oldrule{2.4}
There should be about 3/4 second between each element in the count, with the same amount of time between each of them.
Starters should practice this before the races begin.
Timing of the count is very important for an accurate start.
This count can be in the local language, or a language agreed upon before competition starts.

As an alternative a Startbeep apparatus can be used.
In that case we have a six-count start.
Example: ``beep - beep -beep - beep - beep - buup!''
The interbeep timing is one second.
The first 5 beeps have all the same frequency.
The final tone (buup) has a slightly higher frequency, so that the racer can easily distinguish this tone from the rest.

\oldrule{2.5}
If a heat has to be restarted, the Starter will immediately recall the riders, for example by firing a gun or blowing a whistle or other clear and pre-defined signal.
It is only the earliest false starting rider who gets assigned this false start and might get disqualified.

\section{Finish Line Judge}

\subsection{Judging Finish Line Dismounts}
\oldrule{2.6.1}
One or more officials are required at the finish line to judge dismounts in all races where dismounting is allowed.
These officials must be appointed by the racing referee so they fully understand their crucial job.
The finish line judges are the voice of authority on whether riders must remount and cross the finish line again.
Any riders affected must be clearly and immediately signaled to return to a spot before the finish line, remount without overlapping the finish line, then ride across it again.
The path for backing up may involve going around any finish line timing or optical equipment to prevent data problems for other riders in the race.

\subsection{Timing Penalty For Finish Line Dismounts}
\oldrule{2.6.2}
In electronically timed races, it's possible that no time will be recorded for the rider's successful finish.
Instead of recording an actual finish time, the rider's time will be recorded as 0.01 seconds faster than the next rider to cross the line after their remount and crossing.
If the rider in question is the last one on the track, the time recorded should be their actual time crossing the finish line after their remount.

After the rider has successfully finished the race and there is no correct time for that rider, the rider's finishing time will be calculated based on the time of the next rider to cross the finish line after the rider in question properly finished.
The rider will receive a time penalty which will make his or her time 0.01 second faster than the rider who came after their successful finish.

