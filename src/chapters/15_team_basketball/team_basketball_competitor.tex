\chapter{Competitor Rules}

\section{Safety}

All players must wear shoes as defined in chapter \ref{chap:general_definitions}.

Clothing suggestions for comfort and safety:
\begin{itemize}
\item Short shoelaces, or laces tucked in
\item Definitely no jewelry (watches, necklaces, earrings)
\item No hard clothing or protection, especially above the knees.
\end{itemize}

\section{Unicycles}

Only regular unicycles may be used as defined in chapter \ref{chap:general_definitions}.
The maximum outer diameter of the wheel is 640\,mm (24+ Class).
In addition, the unicycles must not have sharp or protruding parts anywhere that might cause injuries.
Quick-release levers and bolts, for example, must be folded back and not excessively long.
The pedals must be plastic or rubber.
The tire must not leave marks on the floor

\section{Rider Identification}

All players of a team must wear shirts of the same color.
The color must be clearly different from the opponent's color.
At tournaments and other large events each team should have two different colored sets of shirts. In ``A tournaments'' these shirts should be clearly marked with identifying player numbers on them.

\section{Protests}
Protests are handled by the Basketball Director if they concern the global organization of the competition.
Protests concerning a specific game are handled by the referees of this game.

\section{Event Flow}

\subsection{Reminder on Faults and Violations}
A foul is an illegal action involving a contact that can be committed by a player from one team against a player from the opposing team.
If contact occurs beyond what is deemed to be reasonable, or if a player thereby obtains an unfair advantage from it, a foul is committed.
Examples of fouls include pushing, tripping, striking or holding an opposing player and unsportsmanlike conduct.
A foul results in the awarding of the ball to the opposing team and/or free throws.
A violation occurs when the player breaks one of the rules of Basketball.
A violation results in the awarding of the ball to the opposing team.
Examples of violations include traveling, double dribble, backcourt violation, palming the ball, and stepping out of bounds.

\subsection{Mounted Player}
The player can only play the ball while mounted on the unicycle.
A player has established position on the unicycle (``mounted'') when the player has the seat in hand and/or between the legs, with both feet on the pedals, and is not touching anything else for support (including an unmounted player).
Once a player is mounted, the player is considered mounted until some part of their body touches the ground or uses any support, including an unmounted player.
The player throwing the ball inbound must be mounted.

\subsection{Unmounted Player}
If contact is made between the ball and an unmounted player or unicycle, this is a violation and the ball shall be awarded to the other team.
Referees may allow incidental contact between the ball and an unmounted player or unicycle if such contact does not disrupt the flow of the game.
\begin{itemize}
\item An unmounted player must move themself and their unicycle out of the way as soon as possible without disrupting the flow of play.
\item If not possible, the player must leave the unicycle where it lands until it can be retrieved without being disruptive.
\item An unmounted player's unicycle is considered part of the player.
For the purposes of fouls a riderless unicycle that is moving is considered to be out of control.
Thus, if another player is hit by a moving abandoned unicycle, a foul shall be called.
\item If an unmounted player intentionally attempts to play the ball or impede another player, a technical foul shall be called.
\item If a stationary riderless unicycle is disruptive for the opposite team or if it creates a danger for the safety of any player, an ``obstructing unicycle'' violation shall be called.
\end{itemize}

\subsection{Contact of the Ball with a Unicycle}
It is a violation for a player to intentionally strike or stop the ball with any part of their unicycle or leg, however, incidental contact with a player's unicycle or legs is not a violation.
As long as the player is in contact with the unicycle, riding or not, the unicycle is considered part of a player when a ball bounces out of bounds off the unicycle.
If this happens the other team receives possession of the ball.

\subsection{Steps And Traveling}
A traveling violation occurs when a player holding the ball steps in excess of the prescribed limits.
A step is a half revolution of the wheel; meaning that each wheel revolution is the equivalent of two steps because pedaling with one leg only moves the wheel half a revolution.
After a player establishes a pivot foot (see below), the player may not switch the idle foot (in the case of idling) or take a step unless he begins dribbling.

\subsection{Idling, Twisting and Hopping}

Idling is equivalent to the pivot foot and therefore is allowed.
Twisting, where the pedals stay at the same height, the tire stays in contact with the floor, and the rider moves the unicycle left and right is also considered a pivot foot, and therefore allowed.
The player must also stay within a half-meter radius from the point where the idling or twisting started.
A player may not hop (jump up and down repeatedly with the unicycle) while holding the ball.
Hopping while dribbling is permitted.

\subsection{Ball on Floor}
Any player may pick up a ball that is rolling or stopped on the ground.
This can be dangerous, so care must be taken not to foul a player that is bent over to pick up the ball.
If several players simultaneously try to pick up the ball and make contact, the usual rules about fouls apply.
However, if a player not currently trying to pick up the ball is not leaving enough space to a bent player in the referee's judgement, a violation against the disrespectful player may be called.
If the referee judges that no teams are able to pick up the ball, a jump ball situation occurs.
A player may stop a rolling ball with their hand but shall not intentionally make the ball roll on the ground.

\subsection{Four Second Zone}
The three-second zone becomes the four-second zone.

\subsection{Spectacular Tricks}
A player shall only intentionally jump from its unicycle (i.e.\ without trying to stay mounted and to keep the unicycle under control) when attempting to score a goal or to save a ball going out of bounds, and only if it does not create any danger for anyone in the referee's judgement.
Similarly, tricks in which both feet are not on the pedals (like stand up gliding) are only allowed when attempting to score and if it it does not create any danger for anyone in the referee's judgement.
Not respecting these rules will result in a violation being called, and the goal being cancelled (if one is scored in the corresponding action).

\subsection{Screenings}
Screening is only allowed if the player has each arm either bent along the chest or lying along the torso (typically with the hand holding the saddle) and keeps a stable position without changing the orientation of the wheel.
Therefore, being in stand still or hopping while respecting these rules are allowed, while idling and twisting are not (as far as screenings are concerned).
