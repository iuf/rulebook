\chapter{Mountain Unicycling}

For purposes of these rules, Mountain Unicycling (muni) refers to off road races
over any type of terrain. Races can vary from a single heat race with all riders
starting together, to a time- trial type of arrangement with riders going
singly, at intervals. Mountains are not required. Terrain can be anything from
dirt to paved areas, hills, ditches, curbs, rocks, sand, mud, or grass. Unless
otherwise noted, there are no restrictions on wheel size, crank arm length,
brakes or gearing.


\section{General rules}


\subsection{Required Dress}
For all muni events, riders must wear shoes, kneepads, gloves/wristguards and
helmets (definitions, Section 1.2.6). %TODO: update reference 
The IUF allows no exceptions to this for muni events. Additional equipment such
as shin, elbow or ankle protection are optional.

%\subsection{Muni Race Courses}
%Courses must be clearly marked, so the first riders can easily see where to go.
%Unless otherwise noted, non-lane passing rules apply (see Section 2.7.1).
%%TODO: update reference
%A great muni racecourse is often not the same as a great muni
%trail, as there must be room to pass. Though there can be narrow sections, a
%race course must allow enough room for riders to change positions in proportion
%to the amount of riders per heat. For a mass-start race, there must be a wide
%enough starting area, and sufficient length at the beginning for riders to sort
%themselves by speed before the course narrows down. A doubletrack, for instance,
%does not allow passing if there are too many riders in the same place. A section
%of road or wide uphill area can provide space, as well as time to let riders
%spread out. Mass-start events require multiple wide areas to allow passing. All
%courses should strive for a balance of speed, excitement, and safety. When
%course building, it is important to imagine what it would be like for ten or
%more people to fight over it at top speed. Avoid trails that will not work for
%this, or plan for additional heats with smaller numbers of riders. Look for
%areas that will create bottlenecks, such as technical spots and uphill areas,
%and plan the course so these areas do not have too many riders at the same time.

\subsection{Muni Courses}
Courses must be clearly marked, so that riders can easily see where to go. Very
dangerous sections should be secured (E.g. by removing sharp stones / branches
from areas where riders are likely to fall/run into due to the physics of the
course). Unless otherwise noted, non-lane passing rules apply (see Section
2.7.1). %TODO: update reference 
All courses should strive for a balance of speed, excitement, and
safety.\\
For all muni races, every rider must get the chance of at least one test run to
get familiar with the track before the actual race. If possible, the track
should be open for training during all days of the event prior to the race. For
multiday events the muni competitions should take place during the second half
of the event in order to give riders more time to practice on the course.

\subsection{Starting Modes}
The fastest riders should always start first, regardless of the starting mode.
The order can be determined by seeding runs. There are three different types of
starting modes, that can be used in muni races.

\begin{enumerate}
\item Mass start: All riders start at the same time. Mass starts must not be
      used when the race duration is expected to be shorter than 30 minutes. The
      track must provide sufficient space for passing in the first section, so
      that the field of starters is aligned before the track narrows down. Space
      for passing must be given along the track. Mass starts with more than 40
      riders have to be split to avoid accidents.
\item Heats: Groups of riders start at intervals that can vary from 30 seconds
      to a few minutes. The maximum number of riders per heat is determined by
      the average width of the first 100m of the track. There can be one rider
      for each meter in width.
\item Individual starts: Individual riders start at intervals that can vary from
      30 seconds to a few minutes.
\end{enumerate}

\subsection{Age Groups}
% see proposal 90
Age groups must be offered as male and female age group.
The following age groups are the maximum allowable for Muni Competitions:\\
%
\begin{tabular}{ l l}
Under 14 & Youth \\
15-16 & Juniors \\
17-18 & Rookies \\
19-29 & Elites \\
30-50 & Masters \\
50+ & Veterans \\
\end{tabular}


\subsection{IUF muni results}
% see proposal 92
The host must publish two lists of results for each discipline after the
competition: age group based ranking and overall ranking (separating
male/female). There must not be any age group specific restrictions on
equipment.

\subsection{Dismounts And Dismounted Riders}
Dismounts are allowed in all muni races unless otherwise noted. In mass-start
events, dismounted riders must yield to mounted riders behind them as quickly as
possible after a dismount, and until re-mounted. Riders may not impede the
progress of mounted riders when trying to mount. If necessary they must move to
a different location so mounted riders can pass. If riders choose not to ride
difficult sections of the course, they must not pass any mounted riders while
walking or running through them. In time trial-type events, see below for
variations based on the other event details. Violations of these non-riding
rules may result in disqualification or a time penalty, to be determined and
announced before the race start. Riders must also ride completely across the
finish line, as described in Section 2.6. %TODO: update reference

\section{Difficulty Rating for Muni Disciplines}

\subsection{The M Scale}

This scale is solely based on the technical difficulty of a single section. A
section is a part of a trail with a length of 5-20 m. The rating is valid for
good conditions (daylight, dry). It is not influenced by how dangerous the
section is. The following list shows typical features of the different grades.
Not all characteristics have to be satisfied to warrant a respective grade, but
at least two should be fulfilled.

\subsubsection{M0}
pavement or solid soil/compact gravel
no obstacles
slope < 20\%
90° turn within > 2 m and with slope < 10\%

\subsubsection{M1}
partly loose soil/gravel
small obstacles (small stones, flat roots ~ 5 cm)
single 15 cm steps
slope < 40\%
90° turn within > 1 m and with slope < 20\%

\subsubsection{M2}
loose soil/gravel
obstacles (stones, roots ~ 10 cm)
single 30 cm steps
slope < 60\%
90° turn within > 0.5 m and with slope < 30\%

\subsubsection{M3}
loose soil with loose stones (size of few cm)
obstacles (stones, roots ~ 20 cm)
several irregular steps ~ 20 cm each
drops < 1 m, gaps < 0.5 m
slope < 80\%
135° turn within ~ 0.5 m and with slope < 40\%

\subsubsection{M4}
very loose/slippery soil with loose stones (size of several cm)
big obstacles (stones, logs ~ 30 cm)
several irregular steps ~ 30 cm each
drops < 1.5 m, gaps < 1 m
slope < 100\%
135° turn within ~ 0.5 m and with slope < 60\%

\subsubsection{M5}
very loose/slippery soil with loose stones (size of several cm)
very big obstacles (stones, logs ~ 40 cm)
several irregular steps ~ 40 cm each
drops > 1.5 m, gaps > 1 m
slope > 100\%
135° turn within ~ 0.5 m and with slope < 80\%

To refine the scale, grades like M4-5 (a section which has a difficulty between
M4 and M5) are possible. A visual indication for all possible grades is given at
[https://drive.google.com/?authuser=0#folders/0B6zv9bFd15YXdXFOcFR4Z05IU28, to
be replaced by an official link]

\subsection{UDH, UXC: Scales to rate whole trails)}

These scales are based on the features of whole tracks and are important to
define requirements for DH or XC courses at competitions. A certain amount of
points out of a maximum number is given to different criteria. For each
category, the calculation is based on linear interpolation on three intervals.
The interval borders are given in the descriptions below and correspond to 0\%,
20\%, 80\% and 100\% of the maximum points. For values below the lowest interval
border the minimum number of points is given, for values above the highest
interval border the maximum number of points is given. The overall number of
points is the sum of all points of the different criteria. Rounded to the
nearest integer, this number gives the final rating of the track. For measuring
the length, height difference and total ascent of a trail, the geometry of the
trail has to be determined using global navigation satellite systems like GPS
and/or GLONASS. The slope of single sections has to be measured using an
inclinometer. The technical difficulty rating of a single section or of the
whole trail has to follow the definitions of the M scale and should not be
decided by a single person to avoid subjectivity.

UDH Scale (Unicycle Downhill Scale): 0-40

Total Length 0-10: total length of the trail, 0/2/8/10 points given to 0/1/4/8 km

Average Slope 0-10: total height difference divided by total length of the trail, 0/2/8/10 points given to 0/10/25/35%

Maximum Difficulty 0-10: the rating of the most difficult section of the trail using the M scale, 0/2/8/10 points given to M 0/2/3.5/5

Average Difficulty 0-10: the average difficulty rating of the trail using the M scale, 0/2/8/10 points given to M 0/1/2.5/3.5

Excel sheet to calculate the UDH rating: [https://docs.google.com/file/d/0B6zv9bFd15YXZnd4MlBsVy02Y0U/edit, to be replaced by an official link]

UXC Scale (Unicycle Cross Country Scale): 0-40

Total Length 0-15: total length of the trail, 0/3/12/15 points given to 0/10/25/40 km

Total Ascent 0-7.5: total ascent of the trail, 0/1.5/6/7.5 points given to 0/200/800/1500 m

Maximum Slope Uphill 0-5: slope of the steepest uphill section, 0/1/4/5 points given to 0/10/25/35%

Maximum Difficulty 0-5: the rating of the most difficult section of the trail using the M scale, 0/1/4/5 points given to M 0/2/2.5/4

Average Difficulty 0-7.5: the average difficulty rating of the trail using the M scale, 0/1.5/6/7.5 points given to M 0/0.5/1/2

Excel sheet to calculate the UDH rating: [https://docs.google.com/open?id=0B6zv9bFd15YXQlVUV3hQRFRRc3M, to be replaced by an official link]


\section{Disciplines}

\subsection{Uphill Race}
An Uphill muni race challenges riders' ability to climb. Courses may be short
and steep or longer, endurance-related challenges. Generally it is a timed
event, but on an extremely difficult course, riders can be measured as to how
far they ride before dismounting. The race can be offered as a no-dismounts
challenge, which either measures who gets the farthest, or disqualifies anyone
who doesn't complete the distance without a dismount. Multiple tries can be
allowed, or the race can be a simple timed event.

\subsubsection{Dismounted Riders}
If the Uphill race is run as a time trial, riders are intended to ride the
entire distance. In the event of a dismount, the rider must remount the
unicycle: 
%TODO: list formatting
a) At the point where the dismount occurred if the unicycle falls back
down the course toward the start. 
b) Where the unicycle and/or rider come to a
stop after dismounting. Excessive running/walking/stumbling after a dismount may
be grounds for a penalty at the discretion of race of the Referee. 
c) Riders may
also choose to back up (toward the start line) from one of those spots to
remount, if they prefer the terrain there.
 

\subsection{Downhill Race}
A Downhill muni race is a test of speed and ability to handle terrain. Courses
must be primarily downhill but may include flat or uphill sections. Recommended
course length is 2.5 km, or 1 km at a minimum, depending on available terrain,
trails and schedule time. For a course length less than 2 km, two separate runs
should be held. In this case the ranking of the riders is based on the fastest
of the two runs. Riders should race one at a time, released at regular time
intervals. If the schedule has a small time window for the race, riders should
be run in heat sizes that allow passing on the course, and do not bottleneck at
the beginning.

\subsubsection{Dismounted Riders}
Dismounted riders must not impede the progress of, or pass mounted riders. They
must remain aware of riders coming from behind, and not block them with their
unicycles or bodies. Running is not allowed, except momentarily to slow down
after a dismount. Riders may walk if necessary. Riders may receive a time
penalty or be disqualified if they disregard this rule.

\subsection{Cross Country (XC)}
A Cross Country race should be at least 10 km or longer, depending on available
terrain, trails and schedule time. If only shorter trails are available, riders
can be required to complete two or more laps of the course. It is basically any
muni race that is not specifically focused on downhill or uphill. The course can
contain any amount of uphill or downhill riding and is to be about fitness, and
ability to ride fast on rough terrain.

\subsubsection{Dismounted Riders}
If the event is held as a time trial, dismounted rider restrictions must be
announced before the start of the race. Depending on course length and
difficulty, dismounted riders may be required to walk, or walk only limited
distance, or have no restrictions at all.

